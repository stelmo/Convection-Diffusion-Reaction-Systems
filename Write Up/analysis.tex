\documentclass[11pt,fleqn]{article}
\linespread{1.3}
\usepackage{parskip}
\setlength{\parindent}{0pt} %no paragraph indentation
\setlength{\parskip}{2.1ex plus 0.2ex minus 0.2ex} %3x paragraph spacing
\usepackage{geometry}
\geometry{a4paper,left=30mm,right=25mm,top=20mm,bottom=20mm} %margin spacing
\usepackage{fancyhdr}
\pagestyle{fancy}
\fancyhf{}
\renewcommand{\headrulewidth}{0pt}
\rfoot{\thepage}

\usepackage[pdftex]{graphicx} %so that eps files may be included
\usepackage{amsmath}
\usepackage{amssymb}
\usepackage{amsthm}
\usepackage{float}

% Theorems, definitions etc.
\newtheoremstyle{defstyle}
  {10pt} % Space above
  {0pt} % Space below
  {} % Body font
  {} % Indent amount
  {\bfseries} % Theorem head font
  {.} % Punctuation after theorem head
  {0.5em} % Space after theorem head
  {} % Theorem head spec (can be left empty, meaning `normal')
\theoremstyle{defstyle}
\newtheorem{defn}{Definition}[section]
\newtheorem{rmrk}{Remark}[section]
\newtheorem{prop}{Proposition}[section]

\begin{document}

\begin{equation}
-p\partial_{xx}(u) + \tau \partial_x(u) + qu = f
\end{equation}

\begin{equation}
\int_\Omega p\partial_{x}(u)\partial_x(v) + \tau \partial_x(u)v + quv = \int_\Omega fv
\end{equation}

\begin{equation}
a(u,v) = \int_\Omega p\partial_{x}(u)\partial_x(v) + \tau \partial_x(u)v + quv
\end{equation}

\begin{equation}
a(u,v) \neq a(v,u)
\end{equation}

\begin{equation}
a(u,v)  = \int_\Omega p\partial_{x}(u)\partial_x(\bar{v}) + \tau \partial_x(u)\bar{v} + qu\bar{v}
\end{equation}

\begin{equation}
||u-u^h|| \leq ||u-v^h||~~\forall~~v^h \in S^h
\end{equation}


\end{document}